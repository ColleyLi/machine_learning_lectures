\documentclass[aspectratio=169]{beamer}
\mode<presentation>
{
  \usetheme{metropolis}      % or try Darmstadt, Madrid, Warsaw, ...
  \usecolortheme{default} % or try albatross, beaver, crane, ...
  \usefonttheme{structurebold}  % or try serif, structurebold, ...
  \setbeamercolor{background canvas}{bg=white}
  \setbeamertemplate{navigation symbols}{}
  \setbeamertemplate{bibliography item}{\insertbiblabel}
  %\setbeamertemplate{caption}[numbered]
} 
\usepackage[english]{babel}
\usepackage[utf8x]{inputenc}
\usepackage{listings}             % Include the listings-package
\hypersetup{
    colorlinks = true,
    linkcolor = {black},
    urlcolor = {blue}
}

\usepackage{animate}
\usepackage{listings}
\usepackage{bm}

\DeclareMathOperator*{\argmin}{arg\,min}

\title[Deep Learning and Temporal Data Processing]{Deep Learning and Temporal Data Processing}
\subtitle{LSTM in TensorFlow}
\institute{University of Modena and Reggio Emilia}
\author{Andrea Palazzi}
\date{July 17th, 2017}

\def\thisframelogos{}

\newcommand{\framelogo}[1]{\def\thisframelogos{#1}}

\addtobeamertemplate{frametitle}{}{%
\begin{tikzpicture}[remember picture,overlay]
\node[anchor=north east] at (current page.north east) {%
    \foreach \img in \thisframelogos {%
        %\hspace{.5ex}%
        \includegraphics[height=3.5ex]{\img}%
    }%
};
\end{tikzpicture}}

\begin{document}

\framelogo{img/template/logo_unimore_white.png}

\bgroup
\renewcommand{\insertframenumber}{}
\begin{frame}[noframenumbering]
  \titlepage
\end{frame}
\egroup
\begin{frame}{Agenda}
  \tableofcontents
\end{frame}


%%%%%%%%%%%%%%%%%%%%%%%%%%%%%%%%%%%%%%%%%%%%%%%%%%%%%%%%%%%%%%%%%%
%%%%%%%%%%%%%%%%%%%%%%%%%%%%%%%%%%%%%%%%%%%%%%%%%%%%%%%%%%%%%%%%%%
%%%%%%%%%%%%%%%%%%%%%%%%%%%%%%%%%%%%%%%%%%%%%%%%%%%%%%%%%%%%%%%%%%

\section{Synthetic Sequence Dataset}

\begin{frame}[fragile]{Synthetic Sequence Dataset}
For this practice I prepared a \textbf{synthetic dataset} consisting in $\bm{2^{20}}$ \textbf{binary sequences}.\\
\vspace{0.5cm}
You can find it in \texttt{synthetic\_dataset.py}.\\
\vspace{0.5cm}
Loading the data is as simple as:
\begin{verbatim}
from synthetic_dataset import SyntheticSequenceDataset
synthetic_dataset   = SyntheticSequenceDataset()
\end{verbatim}
\end{frame}

%%%%%%%%%%%%%%%%%%%%%%%%%%%%%%%%%%%%%%%%%%%%%%%%%%%%%%%%%%%%%%%%%%
%%%%%%%%%%%%%%%%%%%%%%%%%%%%%%%%%%%%%%%%%%%%%%%%%%%%%%%%%%%%%%%%%%
%%%%%%%%%%%%%%%%%%%%%%%%%%%%%%%%%%%%%%%%%%%%%%%%%%%%%%%%%%%%%%%%%%

\section{Learning to Count}

\begin{frame}[fragile]{Goal}
Our task is to \textbf{count the number of ones in the binary sequences}.\\
\vspace{0.5cm}
The goal of this practice is to implement and train a \textbf{LSTM} \cite{hochreiter1997long} network to do so.
\end{frame}

%%%%%%%%%%%%%%%%%%%%%%%%%%%%%%%%%%%%%%%%%%%%%%%%%%%%%%%%%%%%%%%%%%

\begin{frame}{Useful Functions}
To this purpose, you may find useful the following functions:
\begin{itemize}
\item \texttt{tf.contrib.rnn.LSTMCell}
\item \texttt{tf.nn.dynamic\_rnn}
\item \texttt{tf.transpose}
\item \texttt{tf.gather}
\item \texttt{tf.layers.dense}
\end{itemize}
Please refer to the docs to know the exact API.
\end{frame}

%%%%%%%%%%%%%%%%%%%%%%%%%%%%%%%%%%%%%%%%%%%%%%%%%%%%%%%%%%%%%%%%%%

\begin{frame}{\ }
\centering{
\Large{Good Luck!}
}
\end{frame}

%%%%%%%%%%%%%%%%%%%%%%%%%%%%%%%%%%%%%%%%%%%%%%%%%%%%%%%%%%%%%%%%%%
%%%%%%%%%%%%%%%%%%%%%%%%%%%%%%%%%%%%%%%%%%%%%%%%%%%%%%%%%%%%%%%%%%
%%%%%%%%%%%%%%%%%%%%%%%%%%%%%%%%%%%%%%%%%%%%%%%%%%%%%%%%%%%%%%%%%%

\section{References}

\begin{frame}[t, allowframebreaks]
\frametitle{References}
\bibliographystyle{abbrv}
\bibliography{bibliography}
\end{frame}

\end{document}