\documentclass[aspectratio=169]{beamer}
\mode<presentation>
{
  \usetheme{metropolis}      % or try Darmstadt, Madrid, Warsaw, ...
  \usecolortheme{default} % or try albatross, beaver, crane, ...
  \usefonttheme{structurebold}  % or try serif, structurebold, ...
  \setbeamercolor{background canvas}{bg=white}
  \setbeamertemplate{navigation symbols}{}
  \setbeamertemplate{bibliography item}{\insertbiblabel}
  %\setbeamertemplate{caption}[numbered]
} 
\usepackage[english]{babel}
\usepackage[utf8x]{inputenc}
\usepackage{listings}             % Include the listings-package
\hypersetup{
    colorlinks = true,
    linkcolor = {black},
    urlcolor = {blue}
}

\usepackage{animate}
\usepackage{listings}

\DeclareMathOperator*{\argmin}{arg\,min}

\title[Deep Learning and Temporal Data Processing]{Deep Learning and Temporal Data Processing}
\subtitle{Linear Regression in TensorFlow}
\institute{University of Modena and Reggio Emilia}
\author{Andrea Palazzi}
%\date{June 21th, 2017}

\def\thisframelogos{}

\newcommand{\framelogo}[1]{\def\thisframelogos{#1}}

\addtobeamertemplate{frametitle}{}{%
\begin{tikzpicture}[remember picture,overlay]
\node[anchor=north east] at (current page.north east) {%
    \foreach \img in \thisframelogos {%
        %\hspace{.5ex}%
        \includegraphics[height=3.5ex]{\img}%
    }%
};
\end{tikzpicture}}

\begin{document}

\framelogo{img/template/logo_unimore_white.png}

\bgroup
\renewcommand{\insertframenumber}{}
\begin{frame}[noframenumbering]
  \titlepage
\end{frame}
\egroup
\begin{frame}{Agenda}
  \tableofcontents
\end{frame}


%%%%%%%%%%%%%%%%%%%%%%%%%%%%%%%%%%%%%%%%%%%%%%%%%%%%%%%%%%%%%%%%%%
%%%%%%%%%%%%%%%%%%%%%%%%%%%%%%%%%%%%%%%%%%%%%%%%%%%%%%%%%%%%%%%%%%
%%%%%%%%%%%%%%%%%%%%%%%%%%%%%%%%%%%%%%%%%%%%%%%%%%%%%%%%%%%%%%%%%%

\section{Linear Regression}

\begin{frame}{Goal}
TensorFlow fully unleashes its power when used for deep learning applications. However, this framework can be used to implement any gradient-based learning pipeline.\\
\vspace{0.5cm}
The goal of this first practice session is implementing a simple \textbf{linear regression} model in TensorFlow\cite{tensorflow2015-whitepaper}.
\end{frame}

%%%%%%%%%%%%%%%%%%%%%%%%%%%%%%%%%%%%%%%%%%%%%%%%%%%%%%%%%%%%%%%%%%

\begin{frame}{Data}
For this practice we'll use a small dataset which puts in relation the weight of brain and body for a number of mammal species.\\
\vspace{0.5cm}
The whole dataset can be found here:\\
\url{http://people.sc.fsu.edu/~jburkardt/datasets/regression/x01.txt}\\
\vspace{0.5cm}
The goal is to fit a simple linear regression model (\textit{i.e.} a line) to this data.
\end{frame}

%%%%%%%%%%%%%%%%%%%%%%%%%%%%%%%%%%%%%%%%%%%%%%%%%%%%%%%%%%%%%%%%%%

\begin{frame}{Starter Pack}
I already downloaded the data for you. You'll find it in \texttt{data/brain\_body\_weight.txt}.\\
\vspace{0.5cm}
Also, a function to load the data into python script is available in \texttt{lab\_utils.py}.
\end{frame}

%%%%%%%%%%%%%%%%%%%%%%%%%%%%%%%%%%%%%%%%%%%%%%%%%%%%%%%%%%%%%%%%%%

\begin{frame}{Outline}
Here's the outline of what you're expected to do:
\begin{enumerate}
\item Load data using \texttt{lab\_utils/get\_brain\_body\_data}
\item Define appropriate placeholders using \texttt{tf.placeholder}
\item Define weight and bias variables using \texttt{tf.Variable}
\item Define 1-d linear regression model $y_{pred} = x w + b$
\item Define an appropriate objective function, e.g. $(y_{pred} - y_{true})^2$
\item Create an Optimizer (e.g. \texttt{tf.train.AdamOptimizer})
\item Define a train iteration as one step of loss minimization
\item Loop train iteration until convergence
\end{enumerate}
\end{frame}

%%%%%%%%%%%%%%%%%%%%%%%%%%%%%%%%%%%%%%%%%%%%%%%%%%%%%%%%%%%%%%%%%%

\begin{frame}[fragile, allowframebreaks]{Template}
\begin{verbatim}
    # Read data
    body_weight, brain_weight = # todo
    n_samples = len(body_weight)

    # Define placeholders (1-d)
    x, y = # todo
	
    # Define variables
    w, b = # todo

    # Linear regression model
    y_pred = # todo

    # Define objective function
    loss = # todo

    # Define optimizer
    optimizer = # todo

    # Define one training iteration
    train_step = # todo
    
    with tf.Session() as sess:

        # Initialize all variables
        sess.run(tf.global_variables_initializer())

        for i in range(n_epochs):
            total_loss = 0
            for bo_w, br_w in zip(body_weight, brain_weight):
                _, l = sess.run([train_step, loss],
                                feed_dict={x: bo_w, y: br_w})
                total_loss += l
            print('Epoch {0}: {1}'.format(i, total_loss / n_samples))\end{verbatim}
\end{frame}

%%%%%%%%%%%%%%%%%%%%%%%%%%%%%%%%%%%%%%%%%%%%%%%%%%%%%%%%%%%%%%%%%%
%%%%%%%%%%%%%%%%%%%%%%%%%%%%%%%%%%%%%%%%%%%%%%%%%%%%%%%%%%%%%%%%%%
%%%%%%%%%%%%%%%%%%%%%%%%%%%%%%%%%%%%%%%%%%%%%%%%%%%%%%%%%%%%%%%%%%

\section{References}

\begin{frame}[t, allowframebreaks]
\frametitle{References}
\bibliographystyle{abbrv}
\bibliography{bibliography}
\end{frame}

\end{document}